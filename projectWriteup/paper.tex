\documentclass[12pt]{article}
\usepackage{e-jc}
\usepackage{amsfonts}
\usepackage{amsmath}
\usepackage{amssymb}
\usepackage{amsthm}
\usepackage{graphicx}
\usepackage{wrapfig}
\usepackage{url}
\long\def\symbolfootnote[#1]#2{\begingroup%
\def\thefootnote{\fnsymbol{footnote}}\footnote[#1]{#2}\endgroup}
\newtheorem{lemma}{Lemma}
\newtheorem{theorem}{Theorem}

\title{A Balanced Agent for Capture The Flag Pac-Man}

\author{
Alexander L. Churchill\\
\small \texttt{achur@stanford.edu}\\
\and
Emilio Lopez\\
\small \texttt{elopez1@stanford.edu}\\
\and
Rafael Witten\\
\small \texttt{rwitten@stanford.edu}
}

\date{Mar. 8, 2011}

\begin{document}
\maketitle
\section{Introduction}
Building a balanced Pac Man: the work of champions.

% This is the Bibliography
%%%%%%%%%%%%%%%

\begin{thebibliography}{99}

\bibitem{carlisle} Carlisle, A., Dozier, G. (2001). ``An Off-The-Shelf PSO''.
  {\it Proceedings of the Particle Swarm Optimization Workshop}. pp. 1-6. 

\end{thebibliography}

\end{document}
